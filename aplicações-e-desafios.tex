A aprendizagem automática evolutiva (EML)~\cite{Al-Sahaf2019ALearning} é uma abordagem poderosa que combina técnicas de algoritmos evolutivos e aprendizagem automática para resolver problemas complexos e encontrar soluções ótimas de forma automatizada.

As aplicações atuais de EML são diversas e abrangem campos como agricultura, manufatura, energia, finanças, saúde e outros.
De seguida, encontram-se descritas em maior detalhe algumas dessas aplicações:
\begin{itemize}
    \item Agricultura: planeamento do uso dos terrenos, tomada de decisão na produção de culturas e pesca;
    \item Indústria: redução do tempo e custos de produção e transporte, otimização da cadeia de abastecimento;
    \item Energia: previsão da carga em sistemas de energia, projeção de parques eólicos;
    \item Finanças: análise de dados financeiros, previsão de preços de mercado, análise de risco de falência e gestão de risco de crédito;
    \item Saúde e biomedicina: análise de sequências genéticas, mapeamento genético, previsão e análise de estruturas de ADN, identificação de biomarcadores e cálculo de estruturas proteicas em 3D;
    \item Outras aplicações: previsão de sismos, composição de serviços web, \q{cloud computing}, cibersegurança e jogos.
\end{itemize}

Os Algoritmos Genéticos (GAs) oferecem uma abordagem promissora para otimizar a topologia de redes neuronais numa ampla gama de aplicações.
Segundo Herawan T., et al.~\cite{Chiroma2017NeuralReview}, algumas das áreas de aplicação são:
\begin{itemize}
    \item Classificação e previsão na biologia e na medicina: classificação de cancro de mama, previsão da taxa de oxidação do nitrito, previsão de úlceras de pressão, previsão de intensidade de surtos de carvão/gás;
    \item Reconhecimento de padrões e processamento de imagem: reconhecimento da palma da mão, reconhecimento de aeronaves, processamento de imagens de frutas de cereja;
    \item Previsão e análise de mercado: previsão de preços de ações, previsão de tendências de preços de títulos de segurança e análise de risco de crédito no setor de retalho;
    \item Previsão e otimização de processos: previsão de valores de produção na indústria mecânica e da taxa de oxidação de nanofluidos, parametrização no design de helicópteros, otimização do processo de neutralização de pH;
    \item Otimização de modelos de aprendizagem automática: aproximação de funções, classificação em vários conjuntos de dados, estimativa de parâmetros em modelos ML\@.
\end{itemize}

A otimização de pesos, a seleção de características e, até mesmo, o treino das redes neuronais, são algumas das dimensões onde os GAs podem revelar-se poderosos e versáteis, nomeadamente nas seguintes áreas~\cite{Chiroma2017NeuralReview}:
\begin{itemize}
    \item Previsão e modelação: previsão da precipitação de chuva e saturação dos solos, previsão da doença epilética e da pressão intraocular, previsão do índice de preço de ações, previsão de preços de petróleo bruto, previsão do tempo de congelação e descongelação de alimentos;
    \item Classificação e reconhecimento de padrões: classificação de cancro cervical e de função cerebral cognitiva, reconhecimento de imagem multi-espetral e sonar, reconhecimento de padrões em dados de séries temporais caóticas (eletroencefalograma), classificação de sinais sísmicos;
    \item Otimização de processos, parâmetros e design: design de permutadores de calor e de helicópteros, otimização do processo de endurecimento de plasma, controlo de emissões de óleo de palma, otimização de parâmetros de fermentação e processamento de leite em pó;
    \item Controlo e estabilidade: controlo do equilíbrio bípede, controlo de qualidade em processos industriais, controlo do equilíbrio de robôs autónomos.
\end{itemize}

Apesar das inúmeras aplicações, os algoritmos de inspiração biológica também apresentam alguns desafios, que incluem a seleção de arquiteturas, a falta de resultados de benchmarking, o custo de implementação, a complexidade das aplicações, o tempo de execução razoável, o overfitting em redes neuronais profundas (DNNs), a otimização de hiperparâmetros, a necessidade de alto desempenho de hardware e a falta de flexibilidade das arquiteturas~\cite{Darwish2020ALearning, Mishra2023AAlgorithms}.